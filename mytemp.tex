\documentclass{ltjsarticle}
\usepackage{luatexja}
\usepackage{amsmath, amssymb, type1cm, amsfonts, latexsym, mathtools, bm, amsthm, url, siunitx}
\usepackage{multirow, hyperref, longtable, dcolumn, mathtools, tablefootnote}
\usepackage{tabularx, footmisc, mhchem}
\usepackage{colortbl, here, usebib, microtype}
\usepackage{graphicx}
\usepackage[top = 35.1truemm, bottom = 30truemm, left = 30truemm, right = 30truemm]{geometry}
\usepackage{luatexja-fontspec}
\usepackage{OldStandard}
% フォントを指定するはここから
% \usepackage[T1]{fontenc}
% \renewcommand*\oldstylenums[1]{\textosf{#1}}
% \setmainfont[Ligatures=TeX]{ShipporiMincho-OTF-Regular}
% \setsansfont[Ligatures=TeX]{ShipporiMincho-OTF-Regular}
% \setmainjfont[BoldFont=ShipporiMincho-OTF-Bold]{ShipporiMincho-OTF-Regular}
% \setsansjfont{ShipporiMincho-OTF-Regular}
% \newjfontfamily\jisninety[CJKShape=JIS1990]{ShipporiMincho-OTF-Bold}
% ここまでのコメントを外す
\renewcommand{\r}[1]{\mathrm{#1}}
\renewcommand{\c}{\si{\degreeCelsius}}
\renewcommand{\d}{\r{d}}
\renewcommand{\thefootnote}{\fnsymbol{footnote}}
\renewcommand{\theequation}{\thesection.\arabic{equation}}
\newcommand{\combination}[2]{{}_{#1} \mathrm{C}_{#2}}
\newcommand{\divd}[2]{
  \newcounter{s}
  \newcounter{r}
  \setcounter{s}{0}
  \setcounter{r}{#1}
  \loop
    \ifnum \value{r}>#2
      \addtocounter{s}{1}
      \addtocounter{r}{-#2}
  \repeat
}
\newcounter{y}
\newcounter{mn}
\newcounter{dy}
\newcounter{h}
\newcounter{m}
\divd{\number\time}{60}
\setcounter{y}{\number\year}
\setcounter{mn}{\number\month}
\setcounter{dy}{\number\day}
\setcounter{h}{\value{s}}
\setcounter{m}{\value{r}}
\setcounter{tocdepth}{3}
\mathtoolsset{showonlyrefs = true}
\makeatletter
\@addtoreset{equation}{subsection}
\makeatother
% \bibinput{hogeref}ここにbibtexファイルの名前を入れる.拡張子は書かない
\title{理工学基礎実験レポート}
\date{}
\author{}
\begin{document}
  \maketitle
  \begin{table}[H]
    \begin{center}
      \begin{tabularx}{150mm}{|p{60mm}|X|}
        \hline
        \multirow{2}{*}{実験日} & \multirow{2}{*}{2023年hoge月hoge日 (hoge曜)(午hoge)} \\
        & \\ \hline
        \multirow{3}{*}{実験題目} & \multirow{3}{*}{\large{hoge}} \\
        & \\
        & \\ \hline
      \end{tabularx}
    \end{center}
    \begin{center}
      \begin{tabularx}{150mm}{|X|p{30mm}|X|X|X|X|}
        \hline
        学科 & hoge & クラス & hoge & 学籍番号 & hoge \\ \hline
        報告者氏名 & \multicolumn{5}{c|}{hoge} \\ \hline
      \end{tabularx}
    \end{center}
    \begin{center}
      \begin{tabularx}{150mm}{|p{40mm}|X|X|}
        \hline
        \multirow{4}{*}{共同実験者} &  &  \\ \cline{2-3}
        & &  \\ \cline{2-3}
        & &  \\ \cline{2-3}
        & & \\ \hline
      \end{tabularx}
    \end{center}
    \begin{center}
      \begin{tabularx}{150mm}{|p{60mm}|X|}
        \hline
        レポート提出日 & $\the\value{y}$年$\the\value{mn}$月$\the\value{dy}$日$\the\value{h}$時$\the\value{m}$分 \\ \hline
        レポート再提出日 &  \\ \hline
        &  \\ \hline
      \end{tabularx}
    \end{center}
    \begin{flushleft}
      \begin{tabularx}{75mm}{|X|X|}
        \hline
        室温 & $\ \c$ \\ \hline
        湿度 & $\ \%$ \\ \hline
        気圧 & $\ \r{hPa}$ \\ \hline
      \end{tabularx}
    \end{flushleft}
  \end{table}
  \thispagestyle{empty}
  \addtocounter{page}{-1}
  \clearpage
  % \tableofcontents
  % \thispagestyle{empty}
  % \addtocounter{page}{-1}
  % \clearpage
  \pagestyle{plain}
  \section{目的}
  \section{実験原理}
  \section{実験方法}
  \section{考察}
  \section{結論}
  % ここに参考文献の出力
  % \bibliographystyle{jipsj}
  % \bibliography{hogeref}
\end{document}